\documentclass[a4paper, 12pt]{article}

\usepackage[utf8]{inputenc}
\usepackage[T1]{fontenc}
\usepackage[brazil]{babel}
\usepackage[margin=2.5cm]{geometry}

\usepackage{amsmath}
\usepackage{amssymb}

\usepackage{graphicx}
\usepackage{booktabs}
\usepackage{float}

\usepackage{listings}
\usepackage{xcolor}

\definecolor{codegreen}{rgb}{0,0.6,0}
\definecolor{codegray}{rgb}{0.5,0.5,0.5}
\definecolor{codepurple}{rgb}{0.58,0,0.82}
\definecolor{backcolour}{rgb}{0.95,0.95,0.92}

\lstdefinestyle{mystyle}{
    backgroundcolor=\color{backcolour},   
    commentstyle=\color{codegreen},
    keywordstyle=\color{magenta},
    numberstyle=\tiny\color{codegray},
    stringstyle=\color{codepurple},
    basicstyle=\ttfamily\footnotesize,
    breakatwhitespace=false,         
    breaklines=true,                 
    captionpos=b,                    
    keepspaces=true,                 
    numbers=left,                    
    numbersep=5pt,                  
    showspaces=false,                
    showstringspaces=false,
    showtabs=false,                  
    tabsize=2
}
\lstset{style=mystyle}

\usepackage{hyperref}
\hypersetup{
    colorlinks=true,
    linkcolor=blue,
    filecolor=magenta,      
    urlcolor=cyan,
}

\title{
    \textbf{1º Trabalho Computacional} \\
    \large T10097 - Introdução ao Reconhecimento de Padrões
}
\author{
    \textbf{Aluno:} Hubert Luz de Miranda \\
    \textbf{Matrícula:} 552798
}
\date{Fortaleza, \today}

\begin{document}

\maketitle 

\begin{center}
    \textbf{Professor Responsável:} Prof. Guilherme de Alencar Barreto  \\
    \textbf{Instituição:} Universidade Federal do Ceará (UFC) 
\end{center}

\newpage

\section*{Objetivos}
Os objetivos deste trabalho são:
\begin{enumerate}
    \item Estimar a matriz de covariância de um conjunto de dados, comparando sistematicamente o tempo de execução de diferentes algoritmos.
    \item Avaliar a invertibilidade da matriz de covariância estimada.
    \item Inverter e, se necessário, regularizar as matrizes de covariância obtidas.
\end{enumerate}

\section{Estimativa da Matriz de Covariância}

\subsection{Resultados}
Podemos visualizar os resultados da estimativa da matriz de covariância global no arquivo TC01.ipynb, visualmente os metódos apresentam
resultados iguais, mas ao realizar a subtração entre as matrizes de covariância estimadas e a matriz nativa, notamos que os métodos 1, 3 e 4
apresentam pequenas diferenças (muito próximas de zero), enquanto o método 2 e a função nativa têm diferenças iguais a zero na grande maioria
da matriz de diferenças.

\subsection{Comentários}
Esses resultados nos levam a acreditar que o método nativo utilizado no notebook é implementado de maneira similar ao segundo
método descrito em aula.

\section{Análise de Desempenho dos Métodos}

\subsection{Resultados Numéricos}
\begin{table}[H]
    \centering
    \caption{Tempo de execução (100 rodadas).}
    \label{tab:tempos}
    \begin{tabular}{lcc}
        \toprule
        \textbf{Método} & \textbf{Tempo Médio (s)} & \textbf{Desvio-Padrão (s)} \\
        \midrule
        Método 1        &       &         \\
        Método 2        &       &         \\
        Método 3        &       &         \\
        Método 4        &       &         \\
        Função Nativa   &       &         \\
        \bottomrule
    \end{tabular}
\end{table}

\subsection{Análise Gráfica}


\subsection{Comentários}


\section{Análise de Invertibilidade}

\subsection{Resultados}
\begin{table}[H]
    \centering
    \caption{Análise de invertibilidade das matrizes de covariância.}
    \label{tab:invert}
    \begin{tabular}{lcc}
        \toprule
        \textbf{Matriz de Covariância} & \textbf{Posto (rank)} & \textbf{Número de Condicionamento (rcond)} \\
        \midrule
        Global                         &      &      \\
        Classe 1                       &      &      \\
        Classe 2                       &      &      \\
        \bottomrule
    \end{tabular}
\end{table}

\subsection{Comentários}


\section{Inversão e Regularização das Matrizes}


\subsection{Matrizes Inversas}


\subsection{Técnica de Regularização (se aplicável)}


\subsection{Comentários}



\end{document}